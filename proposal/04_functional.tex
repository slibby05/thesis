
There is a long history of functional programming with many different languages.
However, my concern here is strictly the development of functional compiler technology,
so I will instead focus on the compilers and abstract machines used to run functional languages.

The first real implementation of a functional language was LISP \cite{lisp}.
This was implemented with an interpreter.
While it would late be compiled to more efficient versions \cite{steele78,orbit},
LISP was not considered very practical.

Further implementations would compile a functional language to an abstract machine.
This filled the role of an intermediate representation for functional languages.
Landin's ISWIM language introduced an abstract machine known as the SECD machine \cite{secd}.
This machine was an attempt to define the basic operations of a functional language.
Further improvements were made by Felleisen and Friedman, who produced the CEK machine which is still used in the Racket
programming language \cite{cek}.

While strict functional languages were developing optimizing compilers,
there was a call for lazy functional languages \cite{cons-args}.
The problems raised by lazy evaluation lead to several new approaches to compilers
and intermediate representations.
One of these approaches was to compile to supercombinators,
or functions that contain no free variables.
The most influential version was the TIM (Three Instruction Machine) \cite{tim}.
Although this was easier to implement, it is commonly believed to be inefficient compared to other machines.

The final approach to compiling lazy functional languages is to treat it as a problem in
graph rewriting.
While naive implementations of graph rewriting compilers often resemble interpreters,
there are many optimizations that can be made to improve performance.
The most prominent graph rewriting compilers
is the graph reduction machine, or g-machine \cite{gmachine},
which was used in the Miranda language. 
Two optimizations of the g-machine are removing the spines of functions,
and removing tags identifying nodes in the graphs.
The spine of a function is a sequences of function application
nodes for functions with more than one argument.
These are very common, and add several steps to evaluating any function.
Together the two optimizations resulted in the spineless tagless G-machine,
which is used in the Glasgow Haskell Compiler \cite{functional_PeytonJones, ghc}.

Along with abstract machines, there were several advancements for optimizing programming languages.
Several optimizations, such as constant folding, could be translated immediately to functional languages.
Several more optimizations came from early scheme compilers: Steel's Rabbit compiler, and Kranz' Orbit compiler \cite{steele78,orbit}.
Steel showed how several common constructs in programming languages could be translated into lambda calculus 
\cite{ultimate_imperative, ultimate_declarative},
which allowed for lambda calculus to function as an intermediate language that could be optimized.
Later Steel showed how lambda calculus could be implemented efficiently \cite{ lambda_rename_steel, lambda_goto}.

One of the optimizations in Rabbit and Orbit was the use of Continuation Passing Style as an intermediate representation.
This allowed several dataflow optimizations to be performed more easily, and is similar to SSA 
for imperative languages \cite{continuations_appel}.
Later, Flanagan and Felleisen, among others, refined CPS by removing ``administrative redexes'' to Administrative Normal Form
\cite{anormal_Flanagan}.
This representation is commonly used in modern functional compilers, and is the basis for the STG-machine \cite{stg-peytonJones}.

