
There are traditionally two styles of programming that are considered declarative.
Functional programming uses functions as an abstraction to encourage higher order reasoning.
Logic programming uses relations as an abstraction to encourage relational thinking.
Curry is an attempt to unify both of these approaches as a functional logic language.

Curry is by no means the first language to integrate the ideas of functional and logic programming.
On the logic side, Mercury \cite{mercury} and HAL \cite{hal} included a type system and higher order functions;
Ciao \cite{caio}, a dialect of Prolog, added notation for functions;
OZ \cite{oz} also included support for both functional and logic programming;
and the Toy programming language \cite{toy} adds support for lazy functional logic programming in a Prolog-like syntax.
Curry \cite{Hanus16Curry} is a modern functional logic language with Haskell-like syntax.

Curry is syntactically similar to Haskell, however there are a few differences semantically.
A Curry program is defined by as a collection of function, where each function is given a list of defining equations $lhs = rhs$.
To avoid confusion with equations defined later, and because of their relation to rewrite rules, we refer to these equations as rules.
A partial specification for the syntax of Curry is given below.
While this specification is not complete, it is enough to describe the concepts in this dissertation.

$$\begin{array}{llll}
Program  & \to & (Data\ |\ Function)^* \\
Data     & \to & \text{data}\ name = (name\ Type^*)\ (|\ name\ Type^*)^* \\
Function & \to & name :: Type\ |\ Rule  \\
Rule     & \to & name\ Pattern^* = Expr [\text{where}\ Rule^*] \\
         & |   & name Pattern^* (| Expr = Expr )^* [\text{where}\ Rule^*] \\
Pattern  & \to & constructor\ Pattern^*\ |\ variable\ |\ literal \\
Expr     & \to & Expr\ Expr             & \text{function application}\\
         & |   & name                  & \text{variable or function}\\
         & |   & literal               & \text{number, boolean, or character}\\
         & |   & Expr\ \text{op}\ Expr & \text{binary operation}\\
         & |   & ( Expr )              & \\
         & |   & \text{let}\ Rule\ \text{in}\ Expr  & \text{let expression}
\end{array}$$

Curry has three semantic categories for names in programs.
A \textit{function name} is any name that is defined by $Function$, and a \textit{constructor name}
is any name defined by $Data$.  All other names are variables.
An expression is a \textit{value} if it only contains constructor names or literals.
There is a special function, which we call \texttt{main} that contains no arguments.
We run a Curry program by evaluating the \texttt{main} function.

Those familiar with functional programming may be surprised when learning Curry.
The first surprise is that the order of rules does not matter.
We can define the following function.
\begin{verbatim}
f 1 = 1
f n = 2
\end{verbatim}

While we may reasonably expect the expression \texttt{f 1} to evaluate to \texttt 1, this is not true in Curry.
Curry will find every possible evaluation of an expression.
Since \texttt{f 1} can also match the second rule, the expression is evaluated to both \texttt 1 and \texttt 2.
This is an example of non-deterministic evaluation.
A Curry program will evaluate an expression to every possible value.

There is one important non-deterministic function in Curry.
The ``\texttt{?}'' function will non-deterministically select one of its two operands.
We define the function as

\begin{verbatim}
x ? y = x
x ? y = y
\end{verbatim}

This function allows us to construct non-deterministic expressions without defining new functions.
It will be used repeatedly throughout the rest of this proposal.

Another surprise in Curry is the addition of \textit{free variables}.
A free variable is not given a value in the function, however its value may be constrained.

Consider the function:
\begin{verbatim}
last xs
 | (ys++[y]) == xs = y
  where y, ys free
\end{verbatim}

The variables \texttt{ys} and \texttt y are free, so their value is not determined.
However we made the constraint that \texttt{ys ++ [y]} is equal to the input.
This ensures that \texttt y is the final element of the list.

A final surprise is that expressions are allowed to fail.
An expression fails if it cannot produce a value.
For example \texttt{head []} will fail, because you cannot take the head of an empty list.
If we have a non-deterministic expression \texttt{a ? b} and \texttt a fails, then Curry will only return the result of \texttt b.
We frequently use the behavior to constrain non-deterministic expression by defining equations.
Consider the following rule:
\begin{verbatim}
invert f y
 | f x =:= y = x
   where x free
\end{verbatim}

This \texttt{invert} function asks for all solutions \texttt x such that the equation \texttt{f x == y} is satisfied.
There are many other extensions to Curry including functional patterns and default rules, however
they are not relevant to this thesis.


\subsection{Evaluation}

In order for Curry to be a useful programming language, we must be able to run Curry programs.
The theory of functional logic programming is rooted firmly in term rewriting \cite{Klop92Handbook, Terese03}, 
and graph rewriting \cite{Plump99Handbook, EchahedJanodet97IMAG} in particular.
Rewriting is a purely syntactic system for transforming expressions based entirely on a set of \textit{rewrite rules}.
In Curry, the rewrite rules correspond to the function rules.
As an example, if we have \texttt{reverse [1,2] ++ reverse [3,4]},
then this can rewrite to \texttt{(reverse [2] ++ [1]) ++ reverse [3,4]}.
Rewriting continues until the expression is in a \textit{normal form}, a form in which no more rewriting can be done.
In Curry, normal forms are values or failed computations.

Formally, a Curry expression is a rooted, labeled, directed graph.  Each node has a label and a list of children.
If $e$ is an expression, then a subexpression $s$ of $e$ is a subgraph of $e$ 
such that $s$ has a root, and there is no edge from a node in $s$ to any other node in $e$.
In other words, a subexpression must also be an expression.
A \textit{path} from the root of $e$ to the root of a subexpression $s$ is a sequence of integers such that
the first integer corresponds to a child $c$ of the root of $e$, and the rest of the sequence is a path from $c$ to $s$.
We write $e|_p = s$ to say the there is a path, $p$, from the root of $e$ to the root of $s$.
The notation $e[r]_p$ means: replace the subexpression at position $p$ with expression $r$,
where replacement is defined in the usual way.

A rewrite rule $l \to r$ can be applied to an expression $e$, if there is a subexpression $s$, such that $\sigma(l) = s$
for some substitution $\sigma$.
We write $e\to_{l \to r, p, \sigma} e'$ to mean a rewrite step where we replace the subexpression at $p$ with the $\sigma(r)$.
A \textit{narrowing step} is a rewrite step where we unify \cite{Klop92Handbook} $s$ and $l$ with substitution $\sigma$,
then we replace $\sigma(s)$ with $\sigma(r)$.
We write $e\rightsquigarrow_{l \to r, p, \sigma} e'$ to mean $e$ narrows to $e'$,
with rule $l \to r$, at position $p$, using substitution $\sigma$.

This is a lot of notation, so lets apply it to the reverse example above.
The defining rules for \texttt{reverse} and \texttt{++} are:
\begin{verbatim}
reverse [] = []
reverse (x:xs) = reverse xs ++ [x]

[] ++ ys = ys
(x:xs) ++ ys = x : (xs ++ ys)
\end{verbatim}

Now, if we want to apply a narrowing step to the expression \texttt{reverse [1,2] ++ reverse [3,4]}, we need to find a
subexpression that will unify with a rule.
The only subexpression we can pick is \texttt{reverse [1,2]} to unify with \texttt{reverse (x:xs)}.
This subexpression is at position $1$ of the expression, since it is on the left hand side of \texttt{++}.
Our substitution is $\sigma = \{x \to 1, xs = [2]\}$.
Finally, our rewrite step is 
\texttt{reverse [1,2] ++ reverse [3,4]}
$\rightsquigarrow_{R , p, \sigma}$
\texttt{(reverse [2] ++ [1]) ++ reverse [3,4]}
where $R = reverse\ (x:xs) \to reverse\ xs \mpp [x]$, $p = [1]$, and $\sigma = \{x \to 1, xs \to [2]\}$.

Now with this terminology, evaluating a Curry expression is actually very simple.
First, find a rewrite rule where the left hand side will unify with a subexpression,
then narrow the expression using that rule.
Repeat this process until a normal form is found.

The astute reader may have noticed that I didn't say anything about how we pick the rule.
The process of selecting rules is called the \textit{rewriting strategy}.
We use a needed narrowing rewriting strategy \cite{needed}.
We omit most of the details, because they're long, complicated, and not relevant to an optimizing compiler.

There are two relevant ideas from the needed narrowing strategy.
The first is that needed narrowing is a lazy narrowing strategy.
Curry programs follow the constructor discipline.
The symbols are partitioned into two sets: 
functions symbols which have an associated rewrite rule,
and constructor symbols which do not.
An expression is in head normal form if it is rooted by a constructor.
A lazy narrowing strategy rewrites expressions to head normal form,

The second idea is that we can combine all of the rules for each function into a definitional tree \cite{Antoy92ALP}.
This is a pattern matching tree with some special properties.
Definitional trees are relevant, because the current intermediate representation, FlatCurry, is a textual representation of a
definitional tree.

As an example of a definitional tree, we use the append function from above.

\Tree[.xs++ys [.[]++ys ys ] [.(x:xs)++ys x:(xs++ys) ] ]

While the needed narrowing strategy has been proven optimal in the number of rewrite steps,
there is a lot more we can do to ensure that each rewrite step is computed efficiently.
There has been some attempts to use other compilers to optimize Curry programs.
Unfortunately, this has led to mixed results.
