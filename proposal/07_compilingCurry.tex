
Pakcs, Kics2, and Sprite all have their own strategy for compiling Curry.
While these compilers all compile to different languages,
the more important difference is how they handle non-determinism.
In this section I'll take an in-depth look at these compilers,
and use these ideas to propose an adequate compilation strategy
to demonstrate optimizations.

\subsection{Pakcs}

The Pakcs compiler is built on the foundations of Prolog.
Not only does Pakcs compile Curry to Prolog,
but it does so in such a way that all of the non-standard features 
of Curry are handled by Prolog.
For example, free variables in Curry are translated to free variables in Prolog,
and the choice operator in Curry is translated into parallel clauses in Prolog.
The immediate consequence of this is that all non-determinism is done through backtracking.
While this strategy is efficient, it is not complete.
There are Curry programs that could produce an answer, but will fail to do so in Pakcs.

There are several advantages to translating Curry into Prolog.
One advantage is that the translation is very simple.
A compiler from FlatCurry to Prolog can be written over the course of a week.
Another advantage is the entirely reasonable assumption that the Prolog compiler
will do a better job of implementing non-determinism and free variables than a Curry compiler would.
The basis for this assumption is that more research and work has gone into the Prolog compiler.
Furthermore, any improvements of the efficiency of Prolog directly correlate to improvements
in the efficiency of Curry.
While these benefits are nice, I have already discussed the drawbacks of compiling to Prolog.

I will demonstrate the translation from Curry to Prolog using the \texttt{sum} example from before.

\begin{verbatim}
sum [] = 0
sum (x:xs) = x + xs
\end{verbatim}

Below is a simplified version of the code generated by Pakcs.
For simplicity I am removing code related to residuation \cite{Hanus17PAKCS}.

\begin{verbatim}
sum(Arg,Result) :- hnf(Arg, HArg), sum_1(HArg, Result).

sum_1([],      0).
sum_1([X|XS],  Result)  :- !, hnf(+_F(X, sum_F(XS)), Result).
sum_1(fail(F), fail(F)) :- nonvar(F).
\end{verbatim}

Those familiar with Prolog may find this a little odd.
Why are we calling a mysterious \texttt{hnf} predicate to compute the value of \texttt{X + sum(XS)}?
In fact, why do we have two separate \texttt{sum} predicates at all?
Those familiar with functional interpreters may have already guessed where this is headed.
Pakcs doesn't compile Curry programs, it generates rules that interpret Curry expressions.
There are two rules generated, \texttt{nf} and \texttt{hnf}, which are responsible for reducing expressions
to normal form and head normal form respectively.

The \texttt{nf} rule is actually static across all programs.
It computes the head normal form of the argument,
and computes the normal form of all of the argument's children.
The \texttt{hnf} rule must be generated for every new program.
It is essentially a giant case discriminator for every possible function symbol.
As an example:

\begin{verbatim}
hnf(sum_F(XS), Result)        :- !, sum(XS,Result).
hnf(foldl_F(OP,Z,XS), Result) :- !, foldl(OP,Z,XS, Result).
hnf(++_F(XS,YS), Result)      :- !, ++(XS, YS, Result).
hnf(+_F(X,Y), Result)         :- !, +(X, Y, Result).
\end{verbatim}

Here each function is compiled in a manner similar to \texttt{sum}.
There are a few inherent inefficiencies in this style of compilation.
First, we construct several nodes that we don't need.
For example, if we evaluate the expression \texttt{sum [2]} we will produce the intermediate result.
\texttt{+\_F(2, sum\_F([]))}.
There is no reason to construct the \texttt + node.
Second, every expression that does not directly produce a value
makes a call to the \texttt{hnf} predicate.
This call to \texttt{hnf} is inherently slow,
due to the fact that it contains a giant case discriminator.
In fact, it functions similarly to the giant switch
statement in byte-code interpreters \cite{vmSwitch}.


\subsection{Kics2}

The Kics2 compiler takes a different approach, and compiles Curry to Haskell.
The hope is that deterministic functions can be compiled to Haskell code,
and therefore take advantage of GHC's many optimizations.
Unfortunately, as we have already seen, this doesn't quite work out.

Again, I'll demonstrate Kics2's compilation strategy by example.
Consider the sum function again.
I will simplify the generated code considerably for the sake of readability.
Sum will generate the following Haskell code:

\begin{verbatim}
sum :: OP_List C_Int -> Cover -> ConstStore -> C_Int
sum x1 cd cs = case x1 of
  List -> C_Int 0#
  Cons x2 x3 -> c_plus x2 (sum x3 cd cs) cd cs
  Choice d i l r -> narrow d i (sum l cd cs) (sum r cd cs)
  Fail_List d info -> failCons d (traceFail "sum" [show x1] info)

\end{verbatim}

Here we implement \texttt{sum} as a case statement.
Every algebraic type in Curry is augmented with two possibilities, \texttt{Choice} and \texttt{Fail},
so every function must check those two cases.
If the constructor is the empty list, then we return 0 as normal.
If the constructor is not empty, then we call the \texttt{c\_Plus} function defined in Kics2.
However, if we encounter a choice, we cannot proceed.
In this case we move the choice up one level in our expression, and compute each choice independently.
This is called pull-tabbing \cite{pullTab,Antoy11ICLP}.
As an example the expression \texttt{sum ([1] ?} \texttt{[2])} will match the \texttt{Choice} case, and return 
\texttt{Choice cd i (sum l cd cs) (sum r cd cs)}.
Note that \texttt{(sum l cd cs)} and \texttt{(sum r cd cs)} will not be evaluated due to Haskell's laziness.

There is a non-obvious consequence of this compilation strategy.
Every function that pattern matches in Kics2 contains a case for \texttt{Choice}.
Since \texttt{Choice} is a constructor, it is a head normal form in Haskell.
Therefore, if a choice is ever generated,
it will immediately percolate up to the root of the expression.
This will copy every node above the node in the expression.

\subsection{Sprite}
The final Curry compiler is Sprite \cite{AntoyJost16LOPSTR}.
Sprite compiles Curry to LLVM assembly code, and uses a pull tabbing scheme.
The difference between Sprite and Kics2 is that Sprite uses the Fair Scheme \cite{fair_scheme} for executing choices.
The preliminary results for Sprite look promising.
Unfortunately, since Sprite isn't complete, I can't make any meaningful comparisons to other compilers.

\subsection{Graph Rewriting vs Graph Reduction}
These compilers show the two different ideas of compiling lazy functional programs.
Pakcs compiles all of the Curry functions into rewrite rules.
Expressions are rewritten until they reach a normal form.
The alternative to graph rewriting is graph reduction.
The idea is that the graph knows how to reduce itself to a normal form.
This is the approach commonly taken by lazy functional language compilers.
Each node in the graph contains a pointer to code that will compute a value in head normal form for that node.
These nodes with code pointers are commonly known as closures.

While there is nothing showing that graph rewriting is inherently slower than graph reduction,
the common belief among the functional community is that rewriting is slower.
There is a possibility that, with aggressive optimizations, graph rewriting may be competitive with graph reduction,
but that is not the focus of this research.

\subsection{Compilation}
In order to optimize Curry, I need to have a well founded compilation model.
This model has two main goals: simplicity, and efficiency.
I do not claim that this is the best model for Curry,
but it is easy to understand, and reasonably efficient.

First I need to answer some fundamental questions about this model.
The first question to ask is, ``How should I represent the abstractions in Curry?''
There are two abstractions in Curry that come from logic programming:
non-determinism and free variables.
While I could handle these independently, it is well known that free variables can be converted into
non-determinism and vice versa.
For example, in a language with only free variables I can construct the choice operator as follows:
\begin{verbatim}
(?) :: a -> a -> a
x ? y = if b then x else y
 where b free
\end{verbatim}
Furthermore, if I only have non-determinism,
then I can construct values using a generator for each type:
\begin{verbatim}
freeBool :: Bool
freeBool = True ? False
\end{verbatim}
The expression \texttt{b where b free} could be translated to \texttt{freeBool} by the compiler.
I have no strong reason to prefer one of these representations over the other.
In fact, both of these representations have been proposed \cite{curry_vm, Brassel2011PHD}.
I plan to translate free variables to non-determinism for two reasons:
this is currently the standard representation of Curry programs,
and representing free variables explicitly
will likely require some clever trick such as variant heaps \cite{curry_vm}.
This would complicate the compiler.

\subsection{Non-determinism}
The second task is to determine how to represent non-determinism
This has recently been the subject of a lot of research.
Currently there are four options for representing non-determinism.
Backtracking, Copying, Pull-tabbing, and Bubbling.
All of these options are incomplete in their naive implementations.
However, all of them can be made complete. \cite{fair_scheme}

Backtracking is conceptually the simplest mechanism for non-determinism.
We evaluate an expression normally,
and every time we hit a choice operator, we pick one option.
If we finish the computation, either by producing an answer or failing,
then we undo each of the computations until the last choice expression.
We continue until we've tried every possible choice.

There are a few issues with backtracking.
Aside from being incomplete, a naive backtracking implementation relies on
copying each node as we evaluate it, so we can undo the computation.
Solving incompleteness is a simple matter of using iterative deepening
instead of backtracking.
This poses its own set of issues, such as how do we avoid producing the same answer multiple times,
however these are not difficult problems to solve.
The issue of copying every node we evaluate is a bigger issue, as it
directly competes with any attempt to build an optimizing compiler.
However, I believe that we can avoid creating many of these backtracking nodes.

The following three mechanisms are all based on the idea of copying part of the expression graph.
All of them are incomplete with a naive implementation, however
they can all be made complete using the fair scheme \cite{fair-scheme}.
I'll demonstrate each of these mechanisms with the following expression.
\begin{verbatim}
x + y
 where x
\end{verbatim}

\noindent
This expression has the following graph:
\begin{mdframed}
\centerline{
  \xymatrix@C=-2pt@R=10pt@C=5pt{
      &   & + \bullet \ar@{-}@/_6pt/[ddl] \ar@{-}@/^6pt/[d] \\
      &   & - \bullet \ar@{-}@/_6pt/[dl] \ar@{-}@/^6pt/[dr] \\
      & \code{?} \bullet \ar@{-}@/_6pt/[dl] \ar@{-}@/^6pt/[dr] & & \llap{$y$\,} \bullet \\
    0 \bullet &                                         & 1 \bullet 
  }
}
\end{mdframed}

Copying is a different take on non-determinism.
The idea is straightforward.
Any time we come to a choice node we make a copy of the graph for each choice.
Then we going the copies with a new choice node.
Below I give an example of what the graph looks like before and after copying for the expression:

\begin{mdframed}
\centerline{
  \xymatrix@C=-2pt@R=10pt@C=5pt{
      &   & + \bullet \ar@{-}@/_6pt/[ddl] \ar@{-}@/^6pt/[d] \\
      &   & - \bullet \ar@{-}@/_6pt/[dl] \ar@{-}@/^6pt/[dr] \\
      & \code{?} \bullet \ar@{-}@/_6pt/[dl] \ar@{-}@/^6pt/[dr] & & \llap{$y$\,} \bullet \\
    0 \bullet &                                         & 1 \bullet 
  }
  \hspace*{8em}
  \xymatrix@C=-2pt@R=10pt@C=5pt{
     & & \code{?} \bullet \ar@{-}@/_6pt/[dl] \ar@{-}@/^6pt/[drr]\\
     & + \bullet \ar@{-}@/_6pt/[ddl] \ar@{-}@/^6pt/[d]  & & & + \bullet \ar@{-}@/_6pt/[ddl] \ar@{-}@/^6pt/[d] \\
     & - \bullet \ar@{-}@/_6pt/[dl] \ar@{-}@/^6pt/[dr]  & & & - \bullet \ar@{-}@/_6pt/[dl] \ar@{-}@/^6pt/[dr] \\
    0 \bullet & & \llap{$y$\,} \bullet & 1 \bullet &  & \llap{$y$\,} \bullet
  }
}
\end{mdframed}

Pull-tabbing is the other extreme for moving non-determinism.
Instead of moving the choice node to the root of the graph, we move the choice node up one level.
\cite{pullTab}
A naive implementation of pull-tabbing isn't even valid, so identifier must be included for each variable
to represent which branch it is on.
There is a significant cost to keeping track of these identifiers.

\begin{mdframed}
\centerline{
  \xymatrix@C=-2pt@R=10pt@C=5pt{
      &   & + \bullet \ar@{-}@/_6pt/[ddl] \ar@{-}@/^6pt/[d] \\
      &   & - \bullet \ar@{-}@/_6pt/[dl] \ar@{-}@/^6pt/[dr] \\
      & \code{?} \bullet \ar@{-}@/_6pt/[dl] \ar@{-}@/^6pt/[dr] & & \llap{$y$\,} \bullet \\
    0 \bullet &                                         & 1 \bullet 
  }
  \hspace*{8em}
  \xymatrix@C=-2pt@R=10pt@C=5pt{
     & & + \bullet \ar@{-}@/_6pt/[dddll] \ar@{-}@/^6pt/[d] \ar@{-}@/^6pt/[dddr]\\
     & & \code{?} \bullet \ar@{-}@/_6pt/[dl] \ar@{-}@/^6pt/[drr]\\
     & - \bullet \ar@{-}@/_6pt/[dl] \ar@{-}@/^6pt/[ddrr]  & & & - \bullet \ar@{-}@/_6pt/[dl] \ar@{-}@/^6pt/[ddl] \\
    0 \bullet & & & 1 \bullet\\
              & & & \llap{$y$\,} \bullet
  }
}
\end{mdframed}

Bubbling is a more sophisticated approach to moving non-determinism.
Instead of moving the choice node to the root, we move it to it's dominator. \cite{AntoyBrownChiang06RTA}
Bubbling is always valid, and we aren't copying the entire graph.
Unfortunately computing dominators at runtime is expensive.
There are strategies of keeping track of the current dominator,\cite{bubbling-dom}
but as of this time, there are no known bubbling implementations.

\begin{mdframed}
\centerline{
  \xymatrix@C=-2pt@R=10pt@C=5pt{
      &   & + \bullet \ar@{-}@/_6pt/[ddl] \ar@{-}@/^6pt/[d] \\
      &   & - \bullet \ar@{-}@/_6pt/[dl] \ar@{-}@/^6pt/[dr] \\
      & \code{?} \bullet \ar@{-}@/_6pt/[dl] \ar@{-}@/^6pt/[dr] & & \llap{$y$\,} \bullet \\
    0 \bullet &                                         & 1 \bullet 
  }
  \hspace*{8em}
  \xymatrix@C=-2pt@R=10pt@C=5pt{
     & & \code{?} \bullet \ar@{-}@/_6pt/[dl] \ar@{-}@/^6pt/[drr]\\
     & + \bullet \ar@{-}@/_6pt/[ddl] \ar@{-}@/^6pt/[d]  & & & + \bullet \ar@{-}@/_6pt/[ddl] \ar@{-}@/^6pt/[d] \\
     & - \bullet \ar@{-}@/_6pt/[dl] \ar@{-}@/^6pt/[ddrr]  & & & - \bullet \ar@{-}@/_6pt/[dl] \ar@{-}@/^6pt/[ddl] \\
    0 \bullet & & & 1 \bullet\\
              & & & \llap{$y$\,} \bullet
  }
}
\end{mdframed}


There is one major issue that I've swept under the rug.
So far every Curry implementation use graph rewriting as opposed to graph reduction.
Kics2 comes the closest, be all non-deterministic expressions are copied all the way to the root.
Graph reduction and non-determinism are both well understood individually,
however there is an issue when trying to combine the two.
As an example, we'll look at graph reduction with backtracking.

In a graph rewriting system backtracking is straightforward.
If node $l$ rewrites to node $r$, then we push $(l,r)$ onto a global backtracking stack.
When it comes time to backtrack we simply pop each $(l,r)$ pair from the stack, and
replace the node $r$ with $l$ until we pop a node where $l$ was a choice.
At this point the graph is in the same state it was before the choice was taken.

While this works great, the whole point of using graph reduction over graph rewriting
is to avoid constructing intermediate nodes when evaluating an expression to head normal form.
If a node isn't constructed, then we can't backtrack to it.
One solution to this problem is to identify a set of nodes that must be copied for backtracking.
It is likely impossible to implement non-determinism without creating and copying some nodes,
but this compiler should aim to minimize that copying.

\subsection{Compiler Pipeline}
Since I am demonstrating the value of optimizations, I can reuse existing pieces of Curry compilers.
Currently, every Curry compiler produces an intermediate representation known as FlatCurry.
A FlatCurry program represents a Curry program,
after the transformation to definitional trees has been made.

FlatCurry is similar to the Core language in GHC.
There are new cases for free variables and non-deterministic expressions.
A partial specification of FlatCurry is given below.


$$\begin{array}{lll}
Prog     & = & Prog\ name\ imports\ Type^*\ Function^*\ Op^* \\
\ldots   &   & \\
Function & = & Func\ name\ arity\ visibility\ Type\ Rule \\
Rule     & = & Rule\ var^*\ Expr \\
Expr     & = & Var\ var \\
         & | & Lit\ literal \\
         & | & Comb\ CombType\ name\ Expr^* \\
         & | & Let\ (var, Expr)^*\ Expr\\
         & | & Free\ var^*\ Expr\\
         & | & Or\ Expr\ Expr \\
         & | & Case\ CaseType\ Expr\ (Pattern^*, Expr)^* \\
CombType & = & FunctionAp\ | ConstructorAp  \\
         & | & PartFuctionAp\ | PartConstructorAp \\
Pattern  & = & Pattern\ name\ var^* \\
         & | & LPattern\ literal \\

\end{array}
$$

While FlatCurry is an intermediate representation, it is not well suited to some optimizations.
An important part of this dissertation will involve developing an IR that is suitable
for optimizing Curry programs.
An appropriate IR should represent the basic operations to evaluate a program.
In imperative languages, IRs are similar to assembly code.
In functional languages, IRs are variants of lambda calculus,
where the fundamental operation is function application.
In logic languages, the WAM uses replacement and choices as its fundamental operations.
While I might want to use a combination of lambda calculus and the WAM, this does not seem like the right abstraction.
Furthermore, the WAM is designed for a backtracking system, so this would not be a general solution.

Instead, I propose an IR based on graph rewriting.
This is not a new idea \cite{graph_ir, dactl, functional_PeytonJones}.
In fact, this was one of the major innovations behind the G-machine.
While the STG-machine is an IR for a graph reduction machine, it is still framed in terms
of lambda calculus.
One possibility is that the IR is a variation on ICurry, an imperative language for representing
curry programs. \cite{AntoyJost16LOPSTR}.
ICurry is fundamentally a language for rewriting graphs,
but it has been developed specifically with Curry programs in mind.
Currently the IR is not able to incorporate all of the optimizations.
However, I believe that this IR is a good starting place for optimizing Curry.

